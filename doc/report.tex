\documentclass[a4paper,12pt]{report}

\usepackage{graphicx}
\usepackage{float}
\usepackage{hyperref}
\usepackage{cleveref}

\title{Embedded Systems and Internet-of-Things \\ - \\ Third Assignment}
\author{Kimi Osti}
\date{\today}

\begin{document}
	
	\maketitle
	\tableofcontents
	
	\chapter{System Requirements}
	The system is a smart IoT-based temperature monitor. In particular, it measures a closed environment's temperature at any given time, and controls a window connected to a motor to properly ventilate the room in case of critical temperatures. It also implements a manual mode, which can be activated via a button on the \textit{Window Controller}, or via a web-based \textit{Operator Dashboard}, that allows an operator to physically control the window opening angle thanks to a potentiometer attached to the \textit{Window Controller}.
	\section{Temperature Monitor}
	It's the main system component. It periodically measures the room's temperature, and communicates it to the \textit{Control Unit}, which is responsible of controlling the other component's behavior according to the valued registered by this subsystem. The frequency of the measurements depends on the state of the system, which is stored in the \textit{Control Unit} and is communicated to this component in real-time. It's connected to the \textit{Control Unit} via the \textit{MQTT} protocol, and must include a LED signaling whether the connection is properly established.
	\section{Window Controller}
	It's the in-place operator interface. It's responsible of physically triggering the window movement, according to the values communicated by the \textit{Control Unit} if the system is in automatic mode, or according to the value controlled by the potentiometer if the system is in manual mode. It has a button to switch between these two modes, and it also has a screen that tells the operator the state of the system at any given time. All necessary info is communicated by the \textit{Control Unit} via Serial communication.
	\section{Operator Dashboard}
	It's a web-based user interface that allows operators to work remotely on the system. It shows a graph representing the current state of the system and a brief history of the last measurements, sided by a statistic showing the average, minimum and maximum values in the last period of time. In addition to this info, it exposes a simple operator interface which allows the user to switch between manual and automatic mode, as well as to restore the system status in case an alarm was triggered. It communicates with the \textit{Control Unit} via the \textit{HTTP} protocol in order to reflect user actions on the actual in-place subsystems.
	\section{Control Unit}
	It's the core of the entire system, and it serves as a mediator between subsystem interactions. Its main function is to track the system state and all info related to the measurements and to the current operating mode (manual or automatic). It also determines the sampling frequency for the \textit{Temperature Monitor} according to the current measure, and the window opening percentage according to the system state (if in automatic mode). It's also responsible of storing all system data, including the last measurements that the Dashboard shows in its graph, and the average, minimum and maximum values that are shown to the operator.
	
	\chapter{System Architecture}
	This system can be structured dividing responsibilities according to the MVC architectural pattern.
	\newline In particular, the Controller behavior is implemented by the \textbf{Control Unit}, while the main View component interfacing towards the user would be the \textit{Operator Dashboard}. The \textit{Temperature Monitor} is fully a Model component, since it samples the real-world status value and shares it with the \textit{Control Unit}. Finally, the \textit{Window Controller} can be considered to be almost completely Model, with a part of View in the Control Panel, which allows the user to switch mode between manual and automatic, and to control the window angle when in manual mode.
	\section{Temperature Monitor}
	This subsystem's main feature is sampling the room's temperature with a given frequency. It's also responsible for sending that data to the \textit{Control Unit}, so it must include a component handling the subsystem connectivity to the back-end.
	\newline Its detailed behavior can be described analyzing each sub-component with its responsibilities. Each sub-component can be modeled as a Task, and all Tasks run in parallel in the subsystem on which this part of the application is deployed.
		\subsection{Temperature Measuring Task}
		The central Task is the one responsible of actually measuring the temperature. It can be modeled as a Synchronous Finite State Machine.
		\begin{figure}[H]
			\centering{}
			\includegraphics[scale=0.5]{img/temp-monitor/temp-measuring-fsm.png}
			\caption{FSM modeling the temperature measuring task behavior}
			\label{img:temp-monitor/temp-measuring-fsm}
		\end{figure}
		It is clearly modeled - for clarity - how the state transitions in automatic mode are determined by the temperature values. But in reality, all state transitions are demanded by the \textit{Control Unit} according to the values that it receives from all components, since it is the central Controller for the application.
		\newline Also, it is clearly modeled to be network-agnostic. Indeed, this sub-component is meant to measure at all times, and it will be a responsibility of the network-related tasks to retrieved the stored data when it's supposed to be shared.
		\subsection{Connection Monitoring Task}
		This task is the one responsible for monitoring the system connection state. It's supposed to check whether the component is still online, and when it's not, to try and reconnect. It works on two layers: the first one checks whether the system is connected to the Internet, and the higher one checks whether the \textit{MQTT} subscription is still on.
		\begin{figure}[H]
			\centering{}
			\includegraphics[scale=0.5]{img/temp-monitor/connection-monitoring-fsm.png}
			\caption{FSM modeling the connection monitoring task behavior}
			\label{img:temp-monitor/connection-monitoring-fsm}
		\end{figure}
		\subsection{Communication Task}
		This task is responsible of communicating with the \textit{Control Unit} via the \textit{MQTT} connection set up by the \textit{Connection Monitoring Task}. It ensures that data is properly collected, assembled to form a message and sent to the \textit{Control Unit}. On the other hand, it's also responsible of receiving the response messages published by the \textit{Control Unit}, which can dictate the \textit{Temperature Measuring Task} state, and consequently its sampling frequency.
		\begin{figure}[H]
			\centering{}
			\includegraphics[scale=0.5]{img/temp-monitor/communication-fsm.png}
			\caption{FSM modeling the communication task behavior}
			\label{img:temp-monitor/communication-fsm}
		\end{figure}
		It's here clear how this task too is network agnostic: simply, when there is no connection or no active \textit{MQTT} subscription no messages are received - and the system therefore never goes to the \textit{Receiving} state - and the ones that are sent might be lost.
		\subsection{LED Task}
		This task's main responsibility is to represent to the user whether the sub-system is online.
		\begin{figure}[H]
			\centering{}
			\includegraphics[scale=0.5]{img/temp-monitor/led-fsm.png}
			\caption{FSM modeling the LED task behavior}
			\label{img:temp-monitor/led-fsm}
		\end{figure}
		In this scheme, it's obvious how this Task's state depends directly on the state of the \textit{Connection Monitoring Task}. In particular, the LED is shown as green only when the device is fully connected to the Internet and the \textit{MQTT} subscription is running properly.
		\paragraph{Note} the sub-system's connection state is kept internally, and is not communicated to the \textit{Control Unit}. This because, architecturally speaking, the rest of the system is not concerned about this specific sub-system's connection state, since it would be simply cut out of the communication network. So, connection state is here only tracked to help the LED Task show the user the correct information, but the rest of the system - if well decoupled - is required to keep running even if the communication went down (or theoretically speaking, also if there were multiple devices tracking the temperature and connected to the same \textit{MQTT} topic and broker).
	\section{Window Controller}
	This component's main behavior is to actuate the window opening. In doing so, it's connected to the \textit{Control Unit} via \textit{Serial Line} and it exposes a little operator panel to directly manipulate the window, switching between manual or automatic mode and controlling the window opening level, if in manual mode. Its behavior can be divided in Tasks, running concurrently to achieve the global behavior (it will be discussed in further detail how it's not actually concurrent, since it runs on a single-core \textit{Arduino UNO} board).
		\subsection{Window Controlling Task}
		This Task is responsible of actuating the system's behavior on the actual window. Its only duty is to set the window state each period, querying the \textit{Communication Task} to retrieve the current opening value signaled by the \textit{Control Unit}.
		\begin{figure}[H]
			\centering{}
			\includegraphics[scale=0.5]{img/window-controller/window-controlling-fsm.png}
			\caption{FSM modeling the window controlling task behavior}
			\label{img:window-controller/window-controlling-fsm}
		\end{figure}
		It's obvious how this task's behavior is agnostic of the state of the rest of the system. Simply, each period it retrieves the last received opening value (which is managed completely by the \textit{Control Unit} for consistency reasons) and actuates it to the window via the servomotor.
		\subsection{Operator Input Task}
		This Task is responsible of collecting user input, and of communicating it to the \textit{Control Unit} to modify the entire system's state accordingly.
		\begin{figure}[H]
			\centering{}
			\includegraphics[scale=0.5]{img/window-controller/operator-input-fsm.png}
			\caption{FSM modeling the operator input task behavior}
			\label{img:window-controller/operator-input-fsm}
		\end{figure}
		In this scheme it's underlined how this tasks accepts operator input via the potentiometer only in manual mode. For better consistency though, data is here not directly sent to the \textit{Window Controlling Task}, but rather a decoupling level is inserted between the two, centralizing state handling responsibilities in the \textit{Control Unit}.
		\subsection{Operator Output Task}
		This task is responsible of showing the operator some information about the system's current state. Obviously, for the reasons described above, the system's state is described as received from the \textit{Control Unit}, delegating it all data consistency issues.
		\begin{figure}[H]
			\centering{}
			\includegraphics[scale=0.5]{img/window-controller/operator-output-fsm.png}
			\caption{FSM modeling the Operator Output Task}
			\label{img:window-controller/operator-output-fsm}
		\end{figure}
		In this scheme, it's depicted how this task is showing different information according to the state of the system. On the other hand, state transitions are not handled in any way by this task, and the \textit{modeSwitch} is some input received by the \textit{Control Unit}.
		\subsection{Communication Task}
		This task is responsible of communicating via the \textit{Serial Line} with the \textit{Control Unit}. It's a core feature of this sub-system, since correct communication ensures data (and more importantly behavior) consistency when translating computation into actuation.
		\begin{figure}[H]
			\centering{}
			\includegraphics[scale=0.5]{img/window-controller/communication-fsm.png}
			\caption{FSM modeling the communication task behavior}
			\label{img:window-controller/communication-fsm}
		\end{figure}
		The Communication Task checks for new messages periodically, and expects to receive each period a message containing the current opening level, the current mode and the current temperature. On the other hand, it periodically sends a message containing the user input, representing whether the button has been pressed within the last period and the current value input for the window opening level. Obviously, it's a responsibility of the \textit{Control Unit}, as the system's Controller, to handle user input data according to the state (e.g. to discard the current input value if the system is in automatic mode).
	\section{Operator Dashboard}
	The \textit{Operator Dashboard} serves as a remote user interface to control the system. In particular it offers a graphical representation of the system's current state - which includes a graph of temperature history, the current average, maximum and minimum temperature values in the last period of time, the system operating mode and the window opening level. In addition to this, it allows the operator to switch between automatic and manual modes, as well as a button to restore the system after solving an issue which caused an alarm. It's connected to the \textit{Control Unit} via \textit{HTTP}.
	\section{Control Unit}
	This sub-system is the core of the application. It exchanges data with all other sub-systems, and ensures data consistency storing internally all relevant values. Its main responsibility is to bridge between protocols to coordinate all system components, and to actuate the system's behavior on the real world.
	\newline This sub-system itself is made out of various components.
	\begin{figure}[H]
		\centering{}
		\includegraphics[scale=0.5]{img/control-unit-architecture.png}
		\caption{UML class diagram showing the architecture scheme for the Control Unit sub-components}
		\label{img:control-unit-architecture}
	\end{figure}
	In this scheme, the \textit{CentralController} is the one responsible of ensuring proper data consistency, and it cooperates with all the other sub-components to properly communicate with each sub-system.
	\newline The main concern implementing this sub-system is to ensure that data is properly stored and can be safely accessed, preventing race conditions and other concurrency issues.
	
	\chapter{Implementing Choices}
	\section{Temperature Monitor}
	\section{Window Controller}
	\section{Operator Dashboard}
	\section{Control Unit}
	
\end{document}